\documentclass{article}
\usepackage[utf8]{inputenc}
\usepackage[english,italian,ngerman,french]{babel}
\usepackage[T1]{fontenc}
\begin{document}

\section{Terza (3.) lezione, Chi sono?}

[01]- Francesca e Davide Brambilla abitano a Milano, in Via Verdi nove.
\footnote{
Beachten Sie den Gebrauch der Präpositionen (Verhältniswörter): \textbf{
a Milano} \glqq in Mailand\grqq, in Via Verdi \glqq in der Verdi-Straße\grqq.}
\footnote{Der bestimmte Artikel entfällt bei Straßennamen: \textbf{in via
Verdi} \glqq in der Verdi-Straße\grqq.}
\\
{[02]} Davide è avvocato e lavora al tribunale di Milano.
\\
{[03]} Francesca è medico e lavora all'Ospedale di Sesto San Giovanni.
\footnote{Im Italienischem gibt es die
Berufsbezeichnung \glqq Ärztin\grqq nicht; so kann \textbf{
medico} sowohl \glqq Arzt\grqq als auch \glqq Ärztin\grqq bedeuten.}
\\
{[04]} Marco, invece, ha una laurea in chimica, ma non lavora ancora.
\\
{[05]} Va a Milano per un colloquio di lavoro in un'industria chimica.
\\
{[06]}- Dove lavori, Marco?\\
{[07]}- Non lavoro ancora.\\
{[08]}- Lei lavora a Milano, signora Brambilla?\\
{[09]}- Dove abitate?\\
{[10]}- Abitiamo in Via Manzoni, al numero quattro.\\
{[11]}- Loro lavorano a Milano, noi invece, lavoriamo a Roma.\\
{[12]}-  Marco desidera lavorare a Milano.\footnote{\textbf{lavorare, abitare,
    arrivare} sind Verben, die der 1. Konjugationsgruppe angehören (Infinitiv
auf \textbf{-are}.}

\paragraph{Wer sind sie?}

[01]- Francesca und Davide Brambilla wohnen in Mailand, in [der] Verdi-Straße
nein.\\
{[02]} Davide ist Rechtsanwalt und arbeitet beim (am) Gericht von Mailand.\\
{[03]} Francesca ist Ärztin (Arzt) und arbeitet im (am) Sesto-San-Giovanni-
Krankenhaus (von sechster Sankt Johann).\\
{[04]} Marco dagegen hat einen Doktortitel in Chemie, aber er arbeitet noch
nicht.\\
{[05]} Er geht nach Mailand wegen eines Einstellungsgespräch (für ein Gespräch
der Arbeit) in einem chemischen Industrieunternehmen.\\
{[06]}- Wo arbeitest du, Marco?\\
{[07]}- Ich arbeite noch nicht.\\
{[08]}- Sie arbeiten in Mailand, Frau Brambilla?\\
{[09]}- Wo wohnt ihr?\\
{[10]}- Wir wohnen in [der] Manzoni-Straße, (an der) Nummer vier.\\
{[11]}- Sie (3. Person Plural) arbeiten in Mailand; wir dagegen arbeiten in
Rom.\\
{[12]}- Marco möchte in Mailand arbeiten.\\

\paragraph{Setzen Sie die fehlenden Wörter ein!}

Wo wohnt ihr? Dove ........?
    Wir wohnen in [der] Manzoni-Straße.\\
    ..............in......Manzoni.\\
    Wo arbeiten Sie, [gnädige] Frau?\\
    Dove ................, signora?\\
    Ich hätte gern eine Auskunft, bitte.\\
    .........un'informazione, ... ... ..........\\
    Ich bin in Mailand wegen eines Einstellungsgesprächs.\\
    ...... .... Milano per un ............di lavoro.\\
    Ich möchte in einem chemischen Industriewerk arbeiten.\\
    Desidero .......... in ..'............chimica.\\

    Lösungen: abitate, Abitiamo - Via, lavora, Vorrei - per favore, Sono a -
    colloquio, lavorare - un'industria.


\section{Quarta (4.) lezione, L'arrivo a Milano}

{[01]}- Siamo in orario? domanda una signora a Marco.\\
{[02]}- Credo di sì. Ecco la stazione di Milano.\footnote{\textbf{ecco} ist
eine Redewendung, die nur in Ausrufen gebraucht wird: \textbf{Ecco la stazione!
\glqq Hier ist ja der Bahnhof!\grqq}}\\
{[03]}- La signora, quattro valigie e un ombrello.\\
{[04]}- Facchino! Facchino!\\
{[05]}- Che guaio: oggi i facchini sono in sciopero.\footnote{\textbf{Oggi i
facchini sono in sciopero:} Steht ein Adverb (Umstandswort)
am Anfang des Satzes, so steht im Gegensatz zum Deutschen,
das Prädikat (die Satzaussage) nach dem Subjekt (Satzgegenstand) }
{[06]}- Marco aiuta la signora portare le valigie. Porta anche l'ombrello!\\
{[07]}- Grazie mille, signore, Lei è proprio gentile.\\
{[08]}- Il treno è in orario.\\
{[09]}- I treni arrivano a Milano.\\
{[10]}- La signora domanda l'ora.\\
{[11]} Le signore portano le valigie.\\
{[12]}- Il controllore domanda il biglietto.\\
{[13]} I controllori guardano i biglietti.\\
{[14]}- Ecco la stazione di Milano.\\
{[15]} Le stazioni di Milano e di Roma hanno un traffico intenso.\\
{[16]}. Lo scompartimento di Marco è il numero cinque.\\
{[17]} Lo scompartimenti sono pieni.\footnote{Den bestimmten Artikel
    \textbf{lo} verwendet man vor maskulinen Wörtern, die mit einem \textbf{s}
    beginnen, dem ein Konsonant (Mitlaut) folgt, oder die mit \textbf{z}
    beginnen: \textbf{lo zio} \glqq der Onkel\grqq. Er steht ebenso vor vokalisch
    anlautenden Wörtern, wobei dann das \textbf{o} durch ein Apostroph ersetzt
    wird: \textbf{l'operaio} \glqq der Arbeiter\grqq. Im Plural steht in beiden Fällen
    der Artikel \textbf{gli: gli zii, gli operai.}}
{[18]}- La signora non trova l'ombrello.\\
{[19]} Le signore non trovano gli ombrelli.\\


\paragraph{Setzen Sie die fehlenden Wörter ein!}


Marco trägt die Koffer.\\
Marco ....... ............... .\\
Hier [ist] der Bahnhof von Mailand.\\
..... la ................ di Milano.\\
Die Abteile sind voll.\\
...... .................... sono pieni.\\
Wo ist der Schirm?\\
Dov'è..........................?
Der Schaffnet verlangt die Fahrkarten.\\
...... controllore domanda ... biglietti.\\
Sie sind wirklich freundlich, [mein] Herr!
Lei ................... gentile, signore!\\

Lösungen: porta le valigie, Ecco - stazione, Gli scompartimenti, l'ombrello,
Il - i, è proprio.

\section{Quinta (5.) lezione, A Milano}

{[01]}- Marco entra in un bar.\footnote{\textbf{un bar:} ein Cafè,
in dem es meistens keine Tische gibt und man das Bestellte an der Theke zu
sich nimmt. \textbf{Un caffè:} ein Cafè, in dem man sich setzen kann. Leider
gibt es in Italien nicht mehr soviele Cafès dieser Art. Es gibt aber noch
einige bekannte wie das Caffè Greco und das Caffè Doney in Rom oder das Caffè
Florian in Venedig.}
{[02]}- Un caffè ristretti, per favore!\footnote{Das Gegenteil ist ein \textbf{caffè
lungo:} ein dünner Kaffee, der für Deutsche wahrscheinlich noch sehr stark ist
, entspricht dem deutschen \glqq Esspresso\grqq. In einem Land wie Italien, in dem
der Kaffee ein alltägliches soziales Ritual darstellt (v. a. im Süden), gibt es
viele Variationen zum Thema: vom \textbf{caffé macchiato} \glqq befleckter Kaffee\grqq
(ein Kaffee mit einem Tropfen Milch) bis zum \textbf{caffellatte} \glqq Milchkaffee
\grqq, vom \textbf{caffè corretto} \glqq verbesserter Kaffee\grqq (Kaffee mit einem
Schuss Cognac oder Schnaps) bis zum berühmten \textbf{capuuccino}, ein sehr
starker Kaffee, dem man eine Tasse schaumiger Milch zugibt; danach kann man
noch etwas Kakaopulver auf den Schaum streuen.}
{[03]} Scusi, qual'è l'autobus per andare a via Verdi?\\
{[04]}- Il trentasette. Oppure la metropolitiana, linea B, ma è molto affollata
a quest'ora.\\
{[05]}- Il tempo è bello e Marco decide di prendere l'autobus.\footnote{\textbf{
    prendere} \glqq nehmen\grqq, \textbf{scendere} \glqq austeigen\grqq, \textbf{vedere} \glqq
sehen\grqq usw. gehören den Verben der 2. Gruppe an (Infinitiv auf \textbf{-ere}}
{[06]} È una buona idea! L'autobus non è troppo pieno.\\
{[07]} Scende dopo dieci minuti davanti al palazzo dove abitano Francesca e
Davide.\footnote{\textbf{un palazzo di quattro pianii} \glqq ein Gebäude mit vier
Etagen\grqq, aber auch: \textbf{un palazzo del Rinascimento}
\glqq ein Renaissancepalast\grqq.}
{[08]}- Prendi un caffè?\\
{[09]}- No, prendo un cappuccino.\\
{[10]}- Prendete il taxi?\\
{[11]}- No è troppo caro, prendiamo l'autobus.\\
{[12]}- La signora Lori prende il taxi.\\
{[13]} I signori Lori prendono il taxi.\footnote{Vor den Wörtern \textbf{signore,
signora und signorina} steht der bestimmte Artikel auch dann, wenn ihnen der
Familienname folgt: \textbf{La signorina Palumbo non è a casa.} \glqq Fräulein
Palumbo ist nicht zu Hause.\grqq Beachten Sie auch, dass es im Fall eines
Ehepaares immer heißt \textbf{i signori Fioretti} anstatt \textbf{il signore e
la signora Fioretti}}
{[14]}- Questa signora è napoletana.\\
{[15]} Questo signore, invece, è romano.\footnote{Im Italienischem wird kein
Unterschied gemacht zwischen der geographischen Herkunft einer Person: \textbf{
Marco è romano} \glqq Marco kommt aus Rom, wörtlich: \glqq Marco ist römisch\grqq und
einer Sache: \textbf{palazzo romano} \glqq ein römischer Palast\grqq.}

\paragraph{Setzen Sie die fehlenden Wörter ein}
Herr und Frau Lori nehmen das Flugzeug.\\
I ............... Lori...........l'aero.\\
Gnädige Frau, wo steigen Sie aus?\\
Signora, dove .............?\\
Nehmt ihr den Zug? - Nein, wir nehmen das Flugzeug.\\
.........il treno? - ..., .......... l'aero.\\
Nimmst du einen Kaffee? - Nein, ich nehme einen Cappuccino.\\
.........un caffè? - No, prendo un cappuccino.\\
Der Zug kommt pünktlich an.\\
.....treno........in.....orario.\\

Lösungen: signori - prendono, scende, Prendete - No, prendiamo, Prendi -
prendo, Il - arriva - orario.

\section{Sesta (6.) lezione, In Via Verdi}
{[01]}- Via Verdi è una strada molto bella, ma un po' rumerosa.\footnote{
    \textbf{un po'} \glqq ein bißchen\grqq ist die Kurzform von \textbf{un poco} und
wird in der Umgangssprache vorgezogen.}\\
{[02]} L'appartamento di Francesca e Davide è al quinto piano.\\
{[03]} Marco chiama l'ascensore, ma ... l'ascensore è rotto!\\
{[04]}- Accidenti! Com'è pesante questa valigia! Ah, ecco la porta.\\
{[05]}- Benvenuto, Marco, accomodati!\footnote{\textbf{accomodati} bzw.
(\textbf{accomodatevi,} wenn man sich beim Duzen an mehrere
Personen wendet, und \textbf{si accomodi} beim Siezen) ist ein Ausdruck der
Höflichkeit, der sehr häufig gebraucht wird. Je nach Zusammenhang kann er
bedeuten \glqq komm herein\grqq, \glqq setz dich doch\grqq, \glqq mach es dir
bequem\grqq... Sie werden diesen Audruck häufig hören.}\\
{[06]}- Le strade sono affollate. I negozi sono aperti.\\
{[07]} Questo negozio è chiuso.\\
{[08]}- La valigia di Marco è pesante. Le valigie sono pesanti.\\
{[09]}- L'appartamento è grande e comodo. Gli appartamenti sono grandi e
comodi.\\

\paragraph{In der Verdi-Straße}\\

{[01]}- (Die) Verdi-Straße ist eine sehr schöne Straße, aber ein bisschen laut.\\
{[02]} Die Wohnung von Francesca und Davide ist im fünften Stock (auf der
fünften Ebene).\\
{[03]} Marco ruft den Aufzug, aber ... der Aufzug ist kaputt!\\
{[04]}- Verflixt! Wie schwer dieser Koffer ist! Ah, da [ist ja] die Tür.\\
{[05]}- Willkommen, Marco, komm herein!\\
{[06]}- Die Straßen sind überfüllt. Die Geschäfte sind geöffnet.\\
{[07]} Dieses Geschäft ist geschlossen.
{[08]} Der Koffer von Marco ist schwer. Die Koffer sind schwer.\\
{[09]}- Die Wohnung ist groß und gemütlich. Die Wohnungen sind groß und
gemütlich.\\

Lösungen: 1 Questo - italiano 2 negozi - chiusi 3 Le - sono - pesanti 4 Via -
 una - strada di 5 il biglietto - molto caro 6 L - di - grande - comodo

\paragraph{Esercizio: Mettete le parole che mancano.}


1\t Dieser Herr ist Italiener.\\
.................signore è .......... .\\
2\t Um diese Zeit sind die Geschäfte geschlossen.\\
A quest'ora i ................. sono ........... .\\
3\t Die Koffer von Frau Lori sind sehr schwer.\\
....valigie della signora Lori ...... molto ....... .\\
4\t [Die] Verdi-Straße ist eine schöne Straße von Mailand.\\
.....Verdi è .... bella ..... ..... Milano.\\
5\t In Mailand ist die Busfahrkarte nicht sehr teuer.\\
A Milano .. .................. dell'autobus non è ... .......\\
6\t Die Wohnung von Davide ist groß und gemütlich.\\
..'appartamento ... Davide è ....... e ........ .\\


\section{Ottava (8.) lezione, Benvenuto a casa nostra!}
{[01]}- Ciao Marco! Benvenuto a casa nostra!\footnote{\textbf{a casa mia}
\glqq bei mir zu Hause\grqq, \textbf{a casa tua} \glqq bei dir zu Hause\grqq
, \textbf{a casa di Paolo} \glqq bei Paolo\grqq usw. Beachten Sie die Stellung
des Possessivpronomen (des besitzanzeigenden Fürwortes), das in der Regel
immer dem zugehörigen Substantiv nachgestellt ist. Beachten Sie auch die
verschiedenen Bedeutungen des Wortes \textbf{casa}: Im allgemeinen bezeichnet
\textbf{casa} das Zuhause, und - wie Sie in dieser Unterhaltung gesehen haben -
gebraucht man es oft anstelle des Wortes \textbf{appartamento} \glqq Wohnung
\grqq.}\\
{[02]} Sono molto contento di vederti.\\
{[03]}- Anch'io!\footnote{\textbf{anche Carlo} \glqq Carlo auch\grqq;
\textbf{anche noi}\glqq wir auch \grqq usw. \textbf{Anche} geht dem Wort, auf
das es sich bezieht, immer voran}
{[04]}- Ma accomodati ... Questa è la tua camera da letto,\footnote{Mit der
Befehlsform und dem Infinitiv bildet das Personalpronomen ein einziges Wort,
indem es an diese Formen angehängt wird: \textbf{vederti, accomodati}}\\
{[05]} e quella in fondo al corridoio è la stanza da bagno.\footnote{\textbf{
questo/questa} braucht man für Dinge, die sich in der Nähe des Sprechers
befinden; \textbf{quello/quella} verwendet man für Dinge, die sich weiter
entfernt von ihm befinden.}\\
{[06]} Metti qui la tua valigia e vieni con me in cucina, ti preparo un caffè.\\
{[07]}- Che carina la vostra casa!\\
{[08]}- Sì, non è male, è abbastanza comoda.\\
{[09]} Vedi, questo è il nostro soggiorno, e quello è il mio studio.\\
{[10]}- Davide mostra a Marco il suo appartamento.\\
{[11]} Francesca e Davide sono molto contenti: il loro nuovo appartamento
è proprio carino.\\
{[12]}- Questa è la vostra camera da letto, e quello è il vostro bagno.\\
{[13]}- che buono questo caffè!\\

\paragraph{Willkommen bei uns zu Hause! (in unserem Haus)}\\
\\
{[01]}- Hallo, Marco! Willkommen bei uns zu Hause!\\
{[02]} Ich freue mich sehr (ich bin froh), dich zu sehen.\\
{[03]}- Ich auch!\\
{[04]}- Aber komm doch herein ... dies ist dein Schlafzimmer (Zimmer von
Bett),\\
{[05]} und das da am Ende des Flurs ist das Badezimmer.
{[06]} Stell deinen Koffer hierhin, und komm mit mir in die Küche, ich mache (
bereite ) dir einen Kaffee.\\
{[07]}- Wie hübsch euer Haus [ist]!\\
{[08]}- Ja, es ist nicht schlecht, es ist ziemlich (genug) bequem.\\
{[09]} Siehst du, dies ist unser Wohnzimmer (Aufenthalt), und das da ist mein
Arbeitszimmer.\\
{[10]}- Davide zeigt Marco seine Wohnung.\\
{[11]} Francesca und Davide sind sehr froh: Ihre neue Wohnung ist wirklich
hübsch.\\
{[12]}- Dies ist euer Schlafzimmer, und das da ist euer Bad.\\
{[13]}- Wie gut dieser Kaffee[ist]!


\paragraph{Esercizio}


1\t Francesca zeigt Isabella ihre Wohnung.\\
Francesca ..... a Isabella .. .... appartamento.\\
2\t Dies ist unser Schlafzimmer, und das ist mein Arbeitszimmer.\\
......... è .. ....... camera da letto e ....... è .. ..... studio.\\
3\t Stell deinen Koffer hierhin!\\
Metti qui ... ..... valigia!\\
4\t Ihre (3. Person plural) Wohnung ist wirklich hübsch!\\
... ..... appartamento è ........ carino!\\
5\t Die Koffer der Dame sind schwer.\\
... ......... della signora sono ...... .

Lösungen: 1 mostra - il suo 2 Questa - la nostra - quello - il mio 3 la tua
4 Il loro - proprio 5 Le valigie - pesanti


\end{document}
