\documentclass{article}


\begin{document}

\section{Terza (3.) lezione, Chi sono?}

[01]- Francesca e Davide Brambilla abitano a Milano, in Via Verdi nove.
\footnote{
Beachten Sie den Gebrauch der Präpositionen (Verhältniswörter): \textbf{
a Milano} "`in Mailand"´, in Via Verdi "`in der Verdi-Straße"´.}
\footnote{Der bestimmte Artikel entfällt bei Straßennamen: \textbf{in via
Verdi} "`in der Verdi-Straße.}
\\
{[02]} Davide è avvocato e lavora al tribunale di Milano.
\\
{[03]} Francesca è medico e lavora all'Ospedale di Sesto San Giovanni.
\footnote{Im Italienischem gibt es die
Berufsbezeichnung "`Ärztin"´ nicht; so kann \textbf{
medico} sowohl "`Arzt"´ als auch "`Ärztin"´ bedeuten.}
\\
{[04]} Marco, invece, ha una laurea in chimica, ma non lavora ancora.
\\
{[05]} Va a Milano per un colloquio di lavoro in un'industria chimica.
\\
{[06]}- Dove lavori, Marco?\\
{[07]}- Non lavoro ancora.\\
{[08]}- Lei lavora a Milano, signora Brambilla?\\
{[09]}- Dove abitate?\\
{[10]}- Abitiamo in Via Manzoni, al numero quattro.\\
{[11]}- Loro lavorano a Milano, noi invece, lavoriamo a Roma.\\
{[12]}-  Marco desidera lavorare a Milano.\footnote{\textbf{lavorare, abitare,
    arrivare} sind Verben, die der 1. Konjugationsgruppe angehören (Infinitiv
auf \textbf{-are}.}

\paragraph{Wer sind sie?}

[01]- Francesca und Davide Brambilla wohnen in Mailand, in [der] Verdi-Straße
nein.\\
{[02]} Davide ist Rechtsanwalt und arbeitet beim (am) Gericht von Mailand.\\
{[03]} Francesca ist Ärztin (Arzt) und arbeitet im (am) Sesto-San-Giovanni-
Krankenhaus (von sechster Sankt Johann).\\
{[04]} Marco dagegen hat einen Doktortitel in Chemie, aber er arbeitet noch
nicht.\\
{[05]} Er geht nach Mailand wegen eines Einstellungsgespräch (für ein Gespräch
der Arbeit) in einem chemischen Industrieunternehmen.\\
{[06]}- Wo arbeitest du, Marco?\\
{[07]}- Ich arbeite noch nicht.\\
{[08]}- Sie arbeiten in Mailand, Frau Brambilla?\\
{[09]}- Wo wohnt ihr?\\
{[10]}- Wir wohnen in [der] Manzoni-Straße, (an der) Nummer vier.\\
{[11]}- Sie (3. Person Plural) arbeiten in Mailand; wir dagegen arbeiten in
Rom.\\
{[12]}- Marco möchte in Mailand arbeiten.\\

\paragraph{Setzen Sie die fehlenden Wörter ein!}

Wo wohnt ihr? Dove ........?
    Wir wohnen in [der] Manzoni-Straße.\\
    ..............in......Manzoni.\\
    Wo arbeiten Sie, [gnädige] Frau?\\
    Dove ................, signora?\\
    Ich hätte gern eine Auskunft, bitte.\\
    .........un'informazione, ... ... ..........\\
    Ich bin in Mailand wegen eines Einstellungsgesprächs.\\
    ...... .... Milano per un ............di lavoro.\\
    Ich möchte in einem chemischen Industriewerk arbeiten.\\
    Desidero .......... in ..'............chimica.\\

    Lösungen: abitate, Abitiamo - Via, lavora, Vorrei - per favore, Sono a -
    colloquio, lavorare - un'industria.


\section{Quarta (4.) lezione, L'arrivo a Milano}

{[1]}- Siamo in orario? domanda una signora a Marco.\\
{[2]}- Credo di sì. Ecco la stazione di Milano.\footnote{\textbf{ecco} ist
eine Redewendung, die nur in Ausrufen gebraucht wird: \textbf{Ecco la stazione!
"`Hier ist ja der Bahnhof!"´}}\\
{[3]}- La signora, quattro valigie e un ombrello.\\
{[4]}- Facchino! Facchino!\\
{[5]}- Che guaio: oggi i facchini sono in sciopero.\footnote{\textbf{Oggi i
facchini sono in sciopero:} Steht ein Adverb (Umstandswort)
am Anfang des Satzes, so steht im Gegensatz zum Deutschen,
das Prädikat (die Satzaussage) nach dem Subjekt (Satzgegenstand) }
{[6]}- Marco aiuta la signora portare le valigie. Porta anche l'ombrello!\\
{[7]}- Grazie mille, signore, Lei è proprio gentile.\\
{[8]}- Il treno è in orario.\\
{[9]}- I treni arrivano a Milano.\\
{[10]}- La signora domanda l'ora.\\
{[11]} Le signore portano le valigie.\\
{[12]}- Il controllore domanda il biglietto.\\
{[13]} I controllori guardano i biglietti.\\
{[14]}- Ecco la stazione di Milano.\\
{[15]} Le stazioni di Milano e di Roma hanno un traffico intenso.\\
{[16]}. Lo scompartimento di Marco è il numero cinque.\\
{[17]} Lo scompartimenti sono pieni.\footnote{Den bestimmten Artikel
    \textbf{lo} verwendet man vor maskulinen Wörtern, die mit einem \textbf{s}
    beginnen, dem ein Konsonant (Mitlaut) folgt, oder die mit \textbf{z}
    beginnen: \textbf{lo zio} "`der Onkel"´. Er steht ebenso vor vokalisch
    anlautenden Wörtern, wobei dann das \textbf{o} durch ein Apostroph ersetzt
    wird: \textbf{l'operaio} "`der Arbeiter"´. Im Plural steht in beiden Fällen
    der Artikel \textbf{gli: gli zii, gli operai.}}
{[18]}- La signora non trova l'ombrello.\\
{[19]} Le signore non trovano gli ombrelli.\\


\paragraph{Setzen Sie die fehlenden Wörter ein!}


Marco trägt die Koffer.\\
Marco ....... ............... .\\
Hier [ist] der Bahnhof von Mailand.\\
..... la ................ di Milano.\\
Die Abteile sind voll.\\
...... .................... sono pieni.\\
Wo ist der Schirm?\\
Dov'è..........................?
Der Schaffnet verlangt die Fahrkarten.\\
...... controllore domanda ... biglietti.\\
Sie sind wirklich freundlich, [mein] Herr!
Lei ................... gentile, signore!\\

Lösungen: porta le valigie, Ecco - stazione, Gli scompartimenti, l'ombrello,
Il - i, è proprio.

\section{Quinta (5.) lezione, A Milano}

{[1]}- Marco entra in un bar.\footnote{\textbf{un bar:} ein Cafè,
in dem es meistens keine Tische gibt und man das Bestellte an der Theke zu
sich nimmt. \textbf{Un caffè:} ein Cafè, in dem man sich setzen kann. Leider
gibt es in Italien nicht mehr soviele Cafès dieser Art. Es gibt aber noch
einige bekannte wie das Caffè Greco und das Caffè Doney in Rom oder das Caffè
Florian in Venedig.}
{[2]}- Un caffè ristretti, per favore!\footnote{Das Gegenteil ist ein \textbf{caffè
lungo:} ein dünner Kaffee, der für Deutsche wahrscheinlich noch sehr stark ist
, entspricht dem deutschen "`Esspresso"´. In einem Land wie Italien, in dem
der Kaffee ein alltägliches soziales Ritual darstellt (v. a. im Süden), gibt es
viele Variationen zum Thema: vom \textbf{caffé macchiato} "`befleckter Kaffee"´
(ein Kaffee mit einem Tropfen Milch) bis zum \textbf{caffellatte} "`Milchkaffee
"´, vom \textbf{caffè corretto} "`verbesserter Kaffee"´ (Kaffee mit einem
Schuss Cognac oder Schnaps) bis zum berühmten \textbf{capuuccino}, ein sehr
starker Kaffee, dem man eine Tasse schaumiger Milch zugibt; danach kann man
noch etwas Kakaopulver auf den Schaum streuen.}
{[3]} Scusi, qual'è l'autobus per andare a via Verdi?\\
{[4]}- Il trentasette. Oppure la metropolitiana, linea B, ma è molto affollata
a quest'ora.\\
{[5]}- Il tempo è bello e Marco decide di prendere l'autobus.\footnote{\textbf{
    prendere} "`nehmen"´, \textbf{scendere} "`austeigen"´, \textbf{vedere} "`
sehen"´ usw. gehören den Verben der 2. Gruppe an (Infinitiv auf \textbf{-ere}}
{[6]} È una buona idea! L'autobus non è troppo pieno.\\
{[7]} Scende dopo dieci minuti davanti al palazzo dove abitano Francesca e
Davide.\footnote{\textbf{un palazzo di quattro pianii} "`ein Gebäude mit vier
Etagen"´, aber auch: \textbf{un palazzo del Rinascimento}
"`ein Renaissancepalast"´.}
{[8]}- Prendi un caffè?\\
{[9]}- No, prendo un cappuccino.\\
{[10]}- Prendete il taxi?\\
{[11]}- No è troppo caro, prendiamo l'autobus.\\
{[12]}- La signora Lori prende il taxi.\\
{[13]} I signori Lori prendono il taxi.\footnote{Vor den Wörtern \textbf{signore,
signora und signorina} steht der bestimmte Artikel auch dann, wenn ihnen der
Familienname folgt: \textbf{La signorina Palumbo non è a casa.} "`Fräulein
Palumbo ist nicht zu Hause."´ Beachten Sie auch, dass es im Fall eines
Ehepaares immer heißt \textbf{i signori Fioretti} anstatt \textbf{il signore e
la signora Fioretti}}
{[14]}- Questa signora è napoletana.\\
{[15]} Questo signore, invece, è romano.\footnote{Im Italienischem wird kein
Unterschied gemacht zwischen der geographischen Herkunft einer Person: \textbf{
Marco è romano} "`Marco kommt aus Rom, wörtlich: "`Marco ist römisch"´ und
einer Sache: \textbf{palazzo romano} "`ein römischer Palast"´.}

\paragraph{Setzen Sie die fehlenden Wörter ein}
Herr und Frau Lori nehmen das Flugzeug.\\
I ............... Lori...........l'aero.\\
Gnädige Frau, wo steigen Sie aus?\\
Signora, dove .............?\\
Nehmt ihr den Zug? - Nein, wir nehmen das Flugzeug.\\
.........il treno? - ..., .......... l'aero.\\
Nimmst du einen Kaffee? - Nein, ich nehme einen Cappuccino.\\
.........un caffè? - No, prendo un cappuccino.\\
Der Zug kommt pünktlich an.\\
.....treno........in.....orario.\\

Lösungen: signori - prendono, scende, Prendete - No, prendiamo, Prendi -
prendo, Il - arriva - orario.



\end{document}
