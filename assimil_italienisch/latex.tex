
\documentclass{article}

\usepackage[utf8]{inputenc}
\usepackage[english,italian,german]{babel}
\author{Luke Smith}
\title{My first LateX Document}
\date{für aktuelles Datum, einfach weglassen}


%Anführungsstriche  \glqq Text\grqq  
\begin{document}

\section{Prima lezione, Al telefono}
- Pronto!
Vorrei parlare con Davide, per favore. \footnote{Im Italienischem werden die Personalpronomina (persönlichen Fürwörter) im Nominativ (1. Fall) nur bei einem Vergleich oder zur Hervorhebung gebraucht:\textbf{Carlo e  di Padova; io, invece, sono di Venezia.} \glqq Carlo ist aus Padua; ich dagegen bin aus Venedig.\grqq Aber:\textbf{Sei di Modena? - No, sono di Parma.} \glqq Bist du aus Modena? - Nein, ich bin aus Parma.\grqq In der Übersetzung steht das Pronomen immer, da im Deutschen keine Möglichkeit besteht, es auszulassen} 

- Davide non è casa. Io sono Francesca. Chi e? \footnote{Ein Verb (Zeitwort) wird verneint, indem ihm das Wort \textbf{non} vorangestellt wird. Beachten Sie den Unterschied zwischen \textbf{non} \glqq nicht\grqq und \textbf{no} \glqq nein\grqq!} Io sono Francesca. Chi e?

- Sono Marco. Ciao Franscesca! Come va?

- Bene, grazie, e tu?

- Benissimo!

- Sei a Milano?\footnote{Um eine Frage zu bilden, genügt bei Entscheidungsfragen das Fragezeichen am Ende des Satzes, also immer, wenn die Antwort \glqq ja\grqq oder \glqq nein\grqq lauten muß! D. h., der Aussagesatz selbst wird nicht verändert. Der einzige Unterschied zwischen \textbf{Marco è di Roma.} \glqq Marco ist aus Rom.\grqq und \textbf{Marco è di Roma?} \glqq Ist Marco aus Rom?\grqq besteht in der Satzmelodie. Hören Sie sich die Tonaufnahmen genau an, und versuchen Sie, die Satzmelodie so genau wie möglich nachzuahmen!}

- No, non sono a Milano, sono a Roma. Arrivo a Milano domani mattina.

- Benissimo! A domani, allora.

- D'accordo! Arrivederci!

- Sei di Milano?

- No, non sono di Milano, sono di Roma. Marco è di Roma. Francesca e Davide sono di Milano.\footnote{\textbf{Io sono di Roma; Francesca e Davide sono di Milano.}\glqq Ich bin aus Rom; Francesca und Davide sind aus Mailand.\grqq Die Tatsache, daß man dieselbe Form des Verbs bei der 1. Person Singular (Einzahl) und bei der 3. Person Plural (Mehrzahl) vorfindet, sollte Sie nicht verwirren: Man kann jeweils dem Textzusammenhang entnehmen, ob es sich um \glqq ich bin\grqq oder um \glqq sie sind\grqq handelt.}

- Siete di Firenze?

- No, non siamo di Firenze, siamo di Bologna.

\paragraph{Erste Lektion, Am Telefon}

- Hallo! Ich möchte gern (mit) Davide sprechen, bitte.

- Davide ist nicht zu Hause. Hier ist (ich bin) Francesca. Wer ist [da]?

- Hier ist (ich bin) Marco. Hallo, Francesca. Wie geht['s]?

- Gut, danke, und dir (du)?

- Sehr gut!

- Bist du in Mailand?

- Nein, ich bin nicht in Mailand, ich bin in Rom. Ich komme morgen früh in Mailand an.

- Sehr gut! Bis morgen, also.

- In Ordnung! Auf Wiedersehen!

- Bist du aus (von) Mailand?

- Ich bin nicht aus (von) Mailand, ich bin aus (von) Rom.

- Marco ist aus (von) Rom. Francesca und Davide sind aus (von) Mailand.

- Seid ihr aus (von) Florenz?

- Nein, wir sind nicht aus (von) Florenz, wir sind aus (von) Bologna.

\end{document}
